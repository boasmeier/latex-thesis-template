\chapter*{Demo}
\section*{Cite}
Test cite \apacite{yamauchi_use_2021}.

\section*{Hyperref}
Test Href \href{http://google.ch}{Google}. You can also jump to a chapter or section (see \ref{ch:introduction}). We can also jump to the appendix (see \ref{app:original_project_description}).

\section*{fancyvbr}

\begin{Verbatim}[numbers=left, frame=single]
client = RestClient("localhost:80", "application/json")
students = client.Get("/api/students")
\end{Verbatim}

\section*{color}
Make {\color{red} this} red.

\section*{tcolorbox}
\begin{tcolorbox}
This is a tcolorbox.
\end{tcolorbox}

\section*{Glossary}
The \gls{latex} typesetting markup language is specially suitable 
for documents that include \gls{maths}. 

\section*{Acronym}
This is a \gls{cnn} and the second \gls{cnn}.

\section*{SI units}
\[5 kg m /s^2\]

\[5\ \mathrm{kg}\ \mathrm{m}/\mathrm{s}^2\]

\[5\ \si{kg.m/s^2}\]

\section*{Tables}
\subsection*{Simple Table}
In this thesis you should always use a forward reference for your tables \ref{tab:demo}.
\begin{table}[ht]
\centering
\begin{adjustbox}{max width=\textwidth}
\begin{tblr}{ l|l|l } 
 metric & average & other \\ 
 \hline
 accuracy & 90 \% & 65 \% \\ 
 precision & 81 \% & 70 \% \\
 f1 & 87 \% & 63 \% \\ 
\end{tblr}
\end{adjustbox}
\caption{This is a demo caption for a simple table.}
\label{tab:demo}
\end{table}

The following table \ref{tab:reference_label_prefix} shows which prefix to use for different types of labels.
\begin{table}[ht]
\centering
\begin{adjustbox}{max width=\textwidth}
\begin{tblr}{ l|l l l l l l l }
 Name & Chapter & Section & Subsection & Equation & Figure & Table & Appendix \\
 \hline
 Abbrev. & ch: & sec: & subsec: & eq: & fig: & tab: & app: \\
\end{tblr}
\end{adjustbox}
\caption{Add some letters to the label to recognize what you are referencing.}
\label{tab:reference_label_prefix}
\end{table}


\section*{Figures}
In this thesis you should always use a forward reference for your figures \ref{fig:demo}.
\begin{figure}[ht]
    \centering
    \includegraphics[width=\textwidth,height=\textheight,keepaspectratio]{images/hslu_2022_logo.png}
    \caption{This is a demo caption for a figure.}
    \label{fig:demo}
\end{figure}